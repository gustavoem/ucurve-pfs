\documentclass[12pt, twoside]{report}
\usepackage[utf8]{inputenc}
\usepackage[portuguese]{babel}
\usepackage[a4paper, top=30mm, left=20mm, bottom=20mm,
    right=20mm]{geometry}
\usepackage{graphicx}
\usepackage{fancyhdr}
\fancyhead[LO,RE]{\itshape \nouppercase Chapter \arabic{chapter}}
\usepackage{amssymb}
\usepackage{csquotes}
\usepackage{amsmath}
\usepackage{amsthm}
\usepackage{faktor}
\usepackage[backend=bibtex, 
            maxbibnames=10,
            style=alphabetic]{biblatex}

\pagestyle{fancy}

% Images path
\graphicspath{ {img/} }

% ABNT foreign words should be in italic
\newcommand{\foreignword}[1]{\textit{#1}}
\newcommand{\fieldR}{\mathbb{R}}
\newcommand{\powerset}{\mathcal{P}}
\newcommand{\probability}{\mathbb{P}}
\newcommand{\expectation}{\mathbb{E}}
\newcommand{\algname}[1]{{\tt #1}}

\addbibresource{references.bib}

\newtheorem{mydefinition}{Definição}
\numberwithin{mydefinition}{section}

\title{
    {Projeto de Algoritmos Baseados em Florestas de Posets para o 
     Problema de Otimização U-curve} \\
    {\large Instituto de Matemática e Estatística} \\
}
\author{Gustavo Estrela de Matos}
\date{\today}


\begin{document}


\maketitle

\chapter*{Resumo}
%O problema U-curve é uma formulação de um problema de otimização que 
%pode ser utilizado na etapa de seleção de características em Aprendizado
%de Máquina, com aplicações em desenho de modelos computacionais de 
%sistemas biológicos. Não obstante, soluções propostas até o presente 
%momento para atacar esse problema têm limitações do ponto de vista de 
%consumo de tempo computacional e/ou de memória, o que implica na 
%necessidade do desenvolvimento de novos algoritmos. Nesse sentido, em 
%2012 foi proposto o algoritmo Poset-Forest-Search (PFS), que organiza o
%espaço de busca em florestas de posets. Esse algoritmo foi implementado 
%e testado, com resultados promissores; todavia, novos melhoramentos são
%necessários para que o PFS se torne uma alternativa competitiva para 
%resolver o problema U-curve. Neste projeto propomos a construção de uma 
%versão paralelizada e escalável do algoritmo PFS, utilizando diagramas 
%de decisão binária reduzidos e ordenados. Além disso, propomos adaptar 
%o PFS como um algoritmo de aproximação, no qual o critério de 
%aproximação da solução ótima faça uso do teorema da navalha de Ockham. 
%Os algoritmos desenvolvidos serão implementados e testados em instâncias
%artificiais e também em conjuntos de dados próprios para experimentos 
%comparativos entre diferentes algoritmos de seleção de características.

\tableofcontents

\nocite{*}
\chapter{Introdução}
% - machine learning e seleção de características
%    problema: falta de amostras
%    solução: simplificar o modelo de aprendizado -> seleção de 
%             características    
% - problema de otimização
% - funções de custo
% - aplicações: w-operadores, construção de modelos funcionais
% - algoritmos de seleção de características
% - trabalhos antigos

Seleção de características é uma técnica que pode ser utilizada em uma das
etapas da construção de um modelo de aprendizado de máquina. Ela consiste
em, dado o conjunto de características observadas nas amostras, escolher
um subconjunto que seja ótimo de acordo com alguma métrica. Devemos 
considerar o uso de seleção de características quando a quantidade de
características é muito grande, o que pode tornar o uso do modelo muito caro
do ponto de vista computacional. Outra aplicação dessa técnica é em situações
nas quais a quantidade de amostras é pequena comparada à complexidade do 
modelo original, em outras palavras, quando ocorre sobreajuste (do 
inglês, \foreignword{overfitting}).

Mais formalmente, o problema de seleção de características consiste em
um problema de otimização combinatória em que, dado um conjunto $S$ de 
características, procuramos por um subconjunto $X \in \powerset (S)$
ótimo de acordo com uma função de custo $c : \mathcal{P}(S) \to 
\fieldR_{+}$. {\color{blue}É comum nas abordagens do problema explorar o fato de que
o espaço de busca $\powerset(S)$ junto a relação $\subseteq$ define um
reticulado Booleano {\bf[Adicionar referência(s) para esta afirmação]}}. No geral, a função de custo $c$ deve ser capaz de
medir quão informativas as características $X$ são em respeito ao rótulo
$Y$ do problema de aprendizado; portanto $c$ costuma depender da
estimação da distribuição de probabilidade conjunta de $(X, Y)$.

Quando ocorre a estimação da distribuição de probabilidade conjunta de 
$(X, Y)$, o custo das cadeias do reticulado Booleano reproduzem um
fenômeno conhecido em aprendizado de máquina, o das ``curvas em U''. Para 
entender intuitivamente esse fenômeno, devemos observar que conforme 
subimos uma cadeia do reticulado estamos aumentando o número de 
características sendo consideradas, portanto existem mais possíveis
valores de $X$, permitindo descrever melhor os valores de $Y$; por outro
lado, também precisaríamos de mais amostras para estimar bem 
$\probability (X, Y)$, e, quando isso não é possível, erros de estimação
fazem com que $c(X)$, isto é, o custo de $X$, aumente.

Podemos então considerar um caso particular do problema de seleção de
características em que a função de custo descreve ``curvas em U''
em todas as cadeias do reticulado Booleano. Esse caso particular é 
conhecido como problema U-curve e existem na literatura algoritmos 
ótimos para esse problema como o \algname {U-Curve Branch and Bound 
(UBB), U-Curve-Search (UCS) e Poset Forest Search (PFS)} {\color{blue}[Adicionar referências para os algoritmos]}. A solução do 
problema U-curve tem aplicações em problemas de aprendizado de máquina tais como como projeto
de W-operadores~\cite{MJCJJB} e preditores na estimação de Redes Gênicas 
Probabilísticas~\cite{BCJMJ07}.

O problema U-Curve é NP-difícil~\cite{REI12}; por conta deste fato, os algoritmos 
apresentados até então na literatura têm limitações tanto do ponto de vista de
tempo de computação quanto do uso de memória. Dentre estes algoritmos,
destacamos o PFS, que foi criado como um melhoramento do algoritmo UBB. 
O algoritmo PFS organiza o reticulado Booleano $(\mathcal{P}(S),\subseteq)$ em uma floresta de posets, composta de árvores disjuntas que são subgrafos da árvore que constitui o espaço de busca do algoritmo UBB. Além disso, PFS também mantém uma floresta de posets composta de árvores induzidas a partir de uma árvore construída a partir do reticulado Booleano dual $(\mathcal{P}(S),\supseteq)$. Uma vez que o espaço de busca é composto de várias árvores disjuntas, parece razoável a hipótese de que a paralelização 
desse algoritmo possa trazer ganhos do ponto de vista de consumo de 
tempo. Além disso, a escolha de árvores para etapa de ramificação no 
algoritmo também pode ser explorada e pode trazer ganhos em respeito ao
consumo de tempo e de memória.

{\color{blue}[Comentário: acho que vai ter que revisar a explicação básica da dinâmica do PFS, detalhando-a mais e/ou fazendo referências para minha tese. Uma ideia seria aproveitar o que escrevemos sobre o PFS no projeto de mestrado]}.

\section{Objetivos do Trabalho}
Podemos dividir os objetivos deste trabalho em objetivos gerais e 
específicos.\\

{\bf Objetivos gerais}:
\begin{enumerate}
\item{Criar algoritmos para o problema U-curve que sejam mais eficientes
em consumo de tempo e/ou de memória do que as presentes soluções;}
\item{Verificar a qualidade das soluções encontradas no desenvolvimento
de modelos de Aprendizado Computacional.}
\end{enumerate}

{\bf Objetivos específicos}:
\begin{itemize}
\item{Estudar o algoritmo \algname {Poset Forest Search (PFS)};}
\item{Modificar a etapa de ramificação do algoritmo \algname{PFS} e avaliar
as mudanças na dinâmica do algoritmo;}
\item{Paralelizar o algoritmo \algname{PFS}, com as modificações feitas
na etapa de ramificação (se houver melhorias com tal mudança);}
\item{Criar um novo algoritmo, de natureza paralela e facilmente combinável com outros algoritmos, para o problema 
U-Curve (o algoritmo \algname{PUCS});}
\item{Avaliar o consumo de recursos computacionais dos algoritmos 
criados, comparando com os algoritmos já presentes na literatura como
o \algname{UBB};}
\item{Avaliar os conjuntos de características selecionados por cada 
algoritmo na seleção de modelos de aprendizado computacional, usando 
como exemplo conjuntos de dados do repositório \href{https://archive.ics.uci.edu/ml/index.php}{UCI Machine Learning 
Repository.}}
\end{itemize}

\section{Organização do Trabalho}

A fazer, resumo de cada capítulo da monografia.



\chapter{Conceitos Fundamentais}
% Contextualização e conceitos fundamentais
% .1 O problema de seleção de características
% .2 Funções de custo para esse problema
% .3 Redução para o problema de curvas em U

\section{O problema de seleção de características}
A seleção de características é um problema de otimização combinatória 
em que procuramos o melhor subconjunto de um conjunto de características
$S$. O espaço de busca desse problema é o conjunto potência de $S$, 
$\powerset (S)$, que é a coleção de todos os subconjuntos possíveis de
 $S$. A função de custo desse problema é uma função $c : \powerset (S) 
\to \fieldR_{+}$.

\begin{mydefinition}[Problema de seleção de características] Seja $S$
um conjunto de características, finito e não vazio, e $c$ uma função de 
custo. Encontrar $X \in \powerset (S)$ tal que $c (X) \leq c (Y)$,
$\forall Y \in \powerset (S)$.
\end{mydefinition}

O espaço de busca do problema de seleção de características possui uma
relação de ordem parcial definida pela relação $\subseteq$, portanto
este conjunto é {\bf parcialmente ordenado (poset)}.

\begin{mydefinition}
Uma {\bf \em cadeia} do reticulado booleano é uma sequência $X_1$, 
$X_2$, ..., $X_l$ tal que $X_1 \subseteq X_2 \subseteq \dots 
\subseteq X_l$.
\end{mydefinition}


%No contexto de aprendizado de máquina, é comum que as funções de custo
%utilizadas na seleção de característica descrevam curvas próximas do 
%formato de u nas cadeias do reticulado. Esse fenômeno é conhecido em 
%aprendizado e explicaremos como ele ocorre na seção 
%\ref{fund_concept:cost_functions}.

\section{Funções de custo}
Nesta seção apresentaremos as duas funções de custo mais utilizadas 
durante este trabalho: a entropia condicional média (MCE) e a soma de 
subconjuntos. A primeira foi utilizada na seleção de modelos de
aprendizado, enquanto a segunda foi utilizada para criação e solução
de instâncias artificiais. 
%A função de soma de subconjuntos é 
%decomponível em curvas u e a MCE não, porém explicaremos como a última 
%função deve ter um formato parecido com a da curva em u.

\subsection{Custo de modelos de aprendizado computacional} 
\label{fund_concept:cost_functions} A função de custo utilizada na 
solução do problema deve, de alguma forma, refletir a qualidade do 
conjunto de características avaliado. Por isso,
diferentes aplicações de seleção de características
podem ter diferentes funções de custo. No contexto de aprendizado de 
máquina, uma possível função de custo é a entropia condicional média
(MCE), que já foi utilizada por exemplo na construção de W-operadores
~\cite{MJCJB06}.

\begin{mydefinition}\label{def:conditional_entropy}
Dado um problema de aprendizado em que $Y$ é o conjunto de possíveis
rótulos e $W = (w_1, ..., w_n)$, com $w_i \in A_i$, é o conjunto de
variáveis. Seja $W' = (w_{I(1)}, w_{I(2)}, ..., w_{I(k)})$ um conjunto 
de variáveis (características) escolhidas, $\mathbf{X}$ uma vetor 
aleatório de tamanho $k$ com ${X_j} \in A_{I(j)}$, e $log0 = 0$. Então,
a {\bf \em entropia condicional} de $Y$ dado $\mathbf{X} = \mathbf x$ é:

\begin{center}
$
\begin{aligned}
H (Y | \mathbf{X = x}) = - 
\sum_{y \in Y} \probability (Y = y | \mathbf{X = x}) log \probability (Y = y | \mathbf{X = x})
\end{aligned}
$
\end{center}
\end{mydefinition}

\begin{mydefinition}
Sob o mesmo contexto definido em \ref{def:conditional_entropy}, 
definimos a {\bf \em entropia condicional média} como:
\begin{center}
$
\begin{aligned}
    \expectation[H (Y | \mathbf{X})] = 
    \sum_{\mathbf{x} \in \mathbf{X}} H (Y | \mathbf{X = x}) \probability (\mathbf{X = x})
\end{aligned}
$
\end{center}
\end{mydefinition}

%Vamos usar esta função como exemplo para entender intuitivamente como
%as funções de custo no problema de seleção de características descrevem
%curvas em u.

A função $H$, em teoria da informação, mede o inverso da quantidade 
média de informação que uma variável tem. Esta função atinge valor 
máximo quando a distribuição de probabilidade da variável aleatória em
questão é uniforme (todos valores que ela pode assumir são 
equiprováveis), e tem valores baixos quando essa distribuição é 
concentrada. 

Problemas de aprendizado em que os rótulos tem uma distribuição 
concentrada são mais fáceis do que os problemas em que essa distribuição
é menos concentrada. Tome como exemplo o problema de
classificar o lançamento de uma moeda $\mathbf{x}$ em $y$ (cara ou 
coroa); se toda moeda $\mathbf{x}$ é não viciada, então a distribuição
de $\probability (y | \mathbf{x})$ é pouco concentrada, por outro lado,
quando a moeda é viciada, a distribuição de 
$\probability (y | \mathbf{x})$ é concentrada e é mais fácil 
classificar este problema. Em termos mais formais, o erro do melhor 
classificador do problema mais fácil é menor do que o erro do melhor 
classificador do problema mais difícil.

Portanto, como a função $H$ é capaz de medir a concentração da 
distribuição de $Y$ dado $\mathbf{X = x}$, e quanto maior esta 
concentração mais fácil é o modelo de aprendizado, podemos  dizer
que a função de custo $\expectation[H (Y | \mathbf{X = x})]$ pode 
representar a qualidade do modelo de classificação que usa o conjunto de
características de $\mathbf{X}$.

Agora, como já entendemos o funcionamento da função de custo
MCE e como ela se relaciona com a qualidade do conjunto de 
características avaliado, vamos entender o que acontece no modelo de 
aprendizado e na função de custo que usamos como exemplo quando 
percorremos uma cadeia do reticulado. 

Uma cadeia do poset pode ser vista como uma sequência de possíveis
escolhas de conjuntos de características ao qual a cada passo 
adicionamos uma característica. Isso significa que a cada passo dado
a variável $\mathbf{x}$ ganha uma componente a mais. Quando estamos no 
início da cadeia, poucas variáveis do problema são consideradas, 
portanto há uma grande abstração dos dados dos objetos sendo 
classificados, e conforme subimos uma cadeia, diminuímos a abstração dos
dados e isso faz com que a distribuição de $Y$ dado $\mathbf{x}$ se 
concentre.

Essa concentração da distribuição da probabilidade indica que o custo 
dos subconjuntos deve diminuir conforme subimos por uma cadeia do 
reticulado, ou seja, este raciocínio nos leva a pensar que adicionar 
características sempre melhora a classificação; de fato, o valor de
$\expectation[H (Y | \mathbf{X = x})]$ deve diminuir (até algum ponto 
de saturação) conforme aumentamos o número de variáveis do problema. 
Mas se isso é verdade, por que fazemos seleção de características? A 
inconsistência entre esse raciocínio e a motivação para seleção de 
característica é que essa linha de raciocínio negligenciou que problemas
de classificação (supervisada) dependem de uma amostra da distribuição 
de $Y$ dado $\mathbf{X = x}$, ou seja, não sabemos nem ao menos calcular
$H (Y | \mathbf{X = x})$, podemos apenas estimar o seu valor a partir
da amostra.

A amostra da distribuição de $Y$ dado $\mathbf{X = x}$ é obtida do 
conjunto de treinamento do problema de aprendizado e quando o número
de amostras não é grande o suficiente a qualidade do classificador 
é comprometida. Além disso, o número de amostras necessárias deve
crescer conforme aumentamos a complexidade do modelo de aprendizado 
utilizado. Considerando que quando subimos uma cadeia do reticulado 
booleano estamos aumentando a complexidade do modelo, temos que, a
partir de um certo ponto, a qualidade do classificador que utiliza tal 
conjunto de características deve piorar. 

Portanto, é esperado que a função de custo descreva um formato de u nas 
cadeias do reticulado. No começo da cadeia, o custo deve diminuir por 
conta da maior granularidade dos dados de entrada, até algum ponto onde
a limitação no número de amostras combinada com o aumento da 
complexidade do modelo causem erros de estimação que aumentam o erro
do classificador criado em tal modelo.

No cálculo da entropia condicional média, o efeito do aumento da 
complexidade de $\mathbf X$ é a estimação ruim de 
$\probability (Y = y | \mathbf{X = x})$. Contorna-se este problema 
modificando a entropia condicional média para penalizar a entropia de 
$Y$ quando $\mathbf{x}$ foi observado poucas vezes. A função de custo
utilizada é, então:

\begin{center}
$
\begin{aligned}
    \hat{\expectation}[H (Y | \mathbf{X})] = \frac{N}{t}
    \sum_{\mathbf{x} \in \mathbf{X}} H (Y | \mathbf{X = x}) \probability (\mathbf{X = x})
\end{aligned}
$
\end{center}

\subsection{Soma de subconjuntos}
Para se avaliar o desempenho dos algoritmos criados neste trabalho, 
utilizamos instâncias artificiais que são reduções do problema da soma
de subconjuntos. Este problema consiste em, dado um conjunto finito de
inteiros não-negativos $S$ e um inteiro não-negativo $t$, descobrir se
há um subconjunto de $S$ que soma $t$. Podemos resolver este problema 
com a solução de uma instância do problema de seleção de características
onde o conjunto de características é $S'$ uma cópia de $S$ e a função de
custo é $c$:

\begin{equation*} \label{cost_function:subset_sum}
    c (X) = |t - \sum_{x \in X} x| \text{, para todo } 
                                        X \in \powerset(S') \text{.}
\end{equation*}

Assim como a função de custo MCE, a função de custo de somas de 
subconjuntos também apresenta um formato interessante nas cadeias
do reticulado booleano. Para toda cadeia com elementos $A \subseteq B 
\subseteq C$ vale que $c (B) \leq max\{c (A), c (B)\}$. Vamos provar 
esta propriedade para dois casos disjuntos, quando $|t - \sum_{b \in B}
b| > 0$ e quando $|t - \sum_{b \in B} b| \leq 0$. Começamos a 
demonstração definindo $D = B \setminus A$ e $E = C \setminus B$.

\begin{itemize}
    \item{se $|t - \sum_{b \in B} b| > 0$, então:}
    \begin{align*}
        c (B) & =  |t - \sum_{b \in B} b|  & \\
              & \leq  |t - \sum_{b \in B} b + \sum_{d \in D} d| & 
                \text{(pois $S$ contém apenas números positivos e $t -
                \sum_{b \in B} b > 0$)} \\
              & = |t - \sum_{a \in B \setminus D} a| \\
              & = |t - \sum_{a \in A} a| \\
              & = c (A)
    \end{align*}
    portanto, $c (B) \leq  c (A)$, logo $c (B) \leq max \{c (A), c (C)\}$.
    
    \item{se $|t - \sum_{b \in B} b| \leq 0$, então:}
    \begin{align*}
        c (B) & =  |t - \sum_{b \in B} b|  & \\
              & \leq  |t - \sum_{b \in B} b - \sum_{e \in E} e| & 
                \text{(pois $S$ contém apenas números positivos e $t -
                \sum_{b \in B} b \leq 0$)} \\
              & = |t - \sum_{c \in B \cup E} c| \\
              & = |t - \sum_{c \in C} c| \\
              & = c (C)
    \end{align*}
    portanto, $c (B) \leq  c (C)$, logo $c (B) \leq max \{c (A), c (C)\}$.
\end{itemize}
Como acabamos de provar para os dois casos possíveis, temos que 
$c (B) \leq max \{c (A), c (C)\}$. \qed

\section{O problema U-Curve}
As duas funções de custo apresentadas na seção ~\ref{fund_concept:cost_functions}
descrevem curvas que tem um formato em U (a menos de oscilações) nas 
cadeias do reticulado booleano, vamos definir esta propriedade agora.

\begin{mydefinition}
Uma cadeia é dita {\bf \em maximal} se não existe outra cadeia no 
reticulado que contenha propriamente esta cadeia.
\end{mydefinition}

\begin{mydefinition}\label{fund_concepts:ushape}
Uma função de custo $c$ é dita {\bf \em decomponível em curvas U} se
para toda cadeia maximal $X_1, ..., X_l$, $c(X_j) \leq max \{c (X_i),
c (X_k)\}$ sempre que $X_i \subseteq X_j \subseteq X_k$, $i, j, k \in 
\{1, ..., l\}$.
\end{mydefinition}

Vamos considerar então o problema de seleção de características em que a
função de custo utilizada é decomponível em curvas U. Este é o problema 
central deste trabalho.

\begin{mydefinition}[Problema U-Curve]
Dados um conjunto finito e não-vazio $S$ e uma função de custo $c$ 
decomponível em curvas em U, encontrar um subconjunto $X \in 
\powerset (S)$ tal que $c(X) \leq c(Y)$,  $\forall Y \in \powerset (S)$.
\end{mydefinition}

O problema U-Curve é um caso particular do problema de seleção de 
características com uma propriedade que nos permite achar o mínimo
global sem a necessidade de avaliar cada ponto do reticulado booleano. 
Isso é possível porque a propriedade U-Curve (da decomponibilidade da 
função de custo em curvas U) nos garante que o custo dos elementos de 
uma cadeia não podem cair uma vez que aumentaram. Sejam por exemplo
dois elementos $A \subseteq B$ de $\powerset (S)$, então:
\begin{itemize}
    \item{se $c(B) > c (A)$, então $c (X) > c (A)$ para todo $X$
        do intervalo $[B, \powerset (S)]$;}
    \item{se $c(A) > c (B)$, então $c (X) > c (B)$ para todo $X$ 
        do intervalo $[\emptyset, A]$;}
\end{itemize}

Desta maneira, quando um problema de seleção de características tem uma 
função de custo decomponível em curvas U a menos de algumas oscilações,
é vantajoso aproximar a solução deste problema pela solução encontrada
por um algoritmo de busca do problema U-Curve. Tal abordagem não é 
ótima, porém, como existem poucas oscilações da função de custo, é 
provável que a solução encontrada ainda seja próxima da melhor solução.



\chapter{O algoritmo Parallel U-Curve Search}
O algoritmo \algname{Parallel U-Curve Search} (PUCS) foi desenvolvido 
para resolver o problema U-Curve particionando o espaço de busca em 
partes que podem ser resolvidas independentemente e de forma paralela. 
Além disso, a dinâmica desse algoritmo depende de parâmetros que 
determinam o tempo de execução e qualidade da solução obtida, permitindo
ao usuário adequar o algoritmo aos recursos computacionais disponíveis. 

\section{Princípios do algoritmo}
 
\subsection{Partição do espaço de busca}
Seja $S$ o conjunto de características do problema em questão. O 
primeiro passo do particionamento é escolher arbitrariamente $S'$ um 
subconjunto de $S$; de maneira complementar, definimos 
$\overline{S'} = S \setminus S'$. Agora, sejam $X, Y \in \powerset (S)$
e $\sim$ a relação:

\begin{equation*}
    X \sim Y \iff (X \cap S') = (Y \cap S')
\end{equation*}
Esta relação é de equivalência, pois nela valem:

\begin{itemize}
    \item{reflexividade}
        \begin{align*} 
            X \sim X \text{, pois }
            (X \cap S') = (X \cap S')
        \end{align*} 
    \item{simetria}
        \begin{align*}
            X \sim Y  & \iff \\
            (X \cap S') = (Y \cap S') & \iff \\
            (Y \cap S') = (X \cap S') & \iff \\
            Y \sim X 
        \end{align*}
    \item{transitividade,}
        \begin{align*}
            X \sim Y, Y \sim Z & \Rightarrow \\
            (X \cap S') = (Y \cap S') = (Z \cap S') & \Rightarrow \\
            (X \cap S') = (Z \cap S') & \Rightarrow \\
            X \sim Z
        \end{align*}
\end{itemize}
Portanto, o conjunto das classes de equivalência definidas por $\sim$ é
uma partição do espaço de busca original. Tome como exemplo o conjunto
$S = \{a, b, c\}$; se $S' = {a}$, então existem duas classes de 
equivalência no particionamento do espaço de busca que definimos, 
formados pelos conjuntos $\{\emptyset, b, c, bc\}$ e $\{a, ab, ac, 
abc\}$.

Pela definição da relação $\sim$ temos que a presença de cada 
característica de $S'$ em uma dada parte do reticulado não muda, isto é,
ou ela está presente em todos subconjuntos da parte ou não está presente
em nenhum, portanto, dizemos que estas variáveis são {\bf fixas}. De 
modo análogo, as variáveis de $\overline{S'}$ são {\bf livres}. Tanto 
variáveis fixas quanto livres podem definir reticulados booleanos junto 
a relação de ordem parcial $\subseteq$.

O conjunto $\powerset (S')$ induz um reticulado booleano em que cada
elemento representa uma classe de equivalência do espaço de soluções
do problema original, chamamos este de {\bf reticulado externo}. Para 
cada classe de equivalência (nó do reticulado externo), o conjunto 
$\powerset (\overline {S'})$ induz um outro reticulado booleano ({\bf
reticulado interno}) em que cada elemento representa um subconjunto de
problema original. Seja $A \in \powerset (S')$ um elemento do reticulado 
externo, então cada $B \in \powerset (\overline{S'})$ do reticulado 
interno em $A$ representa o conjunto $X = B \cup A$ do espaço de busca
do problema original.

Os reticulados internos e externo elucidam a estrutura recursiva do 
problema de seleção de características e sugerem que podemos construir 
uma solução ao problema original a partir de soluções de outros 
problemas, sobre os reticulados externo e internos, abordagem conhecida
em computação como divisão e conquista. Seja $\langle S, c \rangle$ uma 
instância do problema de seleção de características, $S'$ o conjunto de 
variáveis fixas, $\overline{S'}$ o conjunto de variáveis livres, e 
$A \in \powerset (S')$ um subconjunto que é nó do reticulado externo, 
então podemos definir um outro problema de seleção de características 
$\langle \overline{S'}, c_{A} \rangle$ em que 
\begin{align*}
    c_{A} (X) = c (X \cup A).
\end{align*}
Resolver a instância $\langle \overline{S'}, c_{A} \rangle$ é 
essencialmente achar o mínimo do problema inicial restrito a classe de
equivalência de $A$, dizemos também que estamos resolvendo a parte $A$. 
Se soubermos em qual classe o mínimo global reside, podemos resolver 
apenas tal parte e garantir que a solução encontrada é a solução do 
problema original.

O algoritmo PUCS percorre o reticulado externo recolhendo partes 
candidatas a conter o mínimo e resolve cada uma destas, escolhendo o 
mínimo das soluções destas partes como o mínimo global do problema.

\subsection{Percorrimento do reticulado externo}


\chapter{Conclusão}
Nesta seção vamos fazer uma revisão deste trabalho, apontando e 
discutindo os resultados obtidos. Também vamos apresentar algumas opções
de linhas de pesquisa para trabalhos futuros relacionados a este.
\section{Revisão do trabalho}
% - estudamos o ubb e o pfs e com isso pudemos
%   - modificar a escolha das raízes
%     - uma escolha arbitrária parece ser melhor do que escolher uma
%       raiz que tem a maior sub-árvore.
%   - modificar estrutura de dados para escolha de raíz
%     - não foi eficiente
%   - paralelizar o PFS
%     - apesar da ramificação ser disjunta, a etapa de atualização
%       compromete o desempenho da paralelização
%   - criar um algoritmo que divide o trabalho
% - usando a ideia de dividir o trabalho, criamos o PUCS
%   - particionamento do espaço em partes iguais - melhor distribuição
%     de trabalho
%   - parâmetros que permitem controlar a granularidade do 
%     particionamento
%   - como algoritmo ótimo é mais lento que outros algoritmos
%   - como heurística acha melhores soluções
% - aplicação em seleção de modelos
%   - diminuimos a quantidade de características mantendo a qualidade
%     dos modelos de aprendizado

Neste trabalho, definimos o problema de seleção de características
e com o exemplo de sua aplicação em seleção de modelos definimos também
o problema U-curve, que foi o problema principal tratado aqui. Estudamos
algoritmos da literatura, investigando e sugerindo modificações a estes,
visando melhorar o seu desempenho computacional. Aplicamos a seleção de
características, feita como uma aproximação pelo problema U-curve, em
conjuntos de dados reais e verificamos como a seleção de características
pode ser benéfica na seleção de modelos de aprendizado.

Estudamos algoritmos baseados em florestas da literatura, \algname{UBB}
e \algname{PFS} e investigamos modificações ao último. Mudamos a 
estratégia de escolha de raízes para ramificação assim como a estrutura
de dados para armazenamento destas raízes no \algname{PFS}, entretanto
estas modificações não implicaram em melhoras significativas no 
desempenho do algoritmo. Sugerimos uma paralelização do algoritmo
\algname{PFS}, justificada pelo fato da etapa de ramificação poder 
acontecer de maneira independente, mas a etapa de atualização, que 
ocorre após a ramificação, mostrou-se dependente e complicada de ser
paralelizada, o que implicou em um desempenho pior do que a versão
sequencial do mesmo algoritmo. 

Ainda no escopo de modificações do \algname{UBB} e \algname{PFS}, 
propomos o \algname{UBB-PFS}, um algoritmo que utiliza o \algname{UBB} 
para decompor o espaço de busca em sub-árvores que são intervalos
do reticulado Booleano. Conseguimos reduzir a solução de cada sub-árvore
a um problema de seleção de características auxilar, em que cada um 
destes pode ser resolvido pelo algoritmo \algname{PFS} de maneira 
independente e paralela. Com a criação de sub-problemas, a paralelização 
deste algoritmo tornou-se mais simples e menos entrelaçada do que na 
outra tentativa de paralelização do \algname{PFS}. Como resultado, o 
novo algoritmo teve tempo de execução e número de chamadas da função 
custo intermediários, com menos nós computados do que o \algname{UBB} e
menor tempo de execução do que o \algname{PFS}. 

A criação de árvores feita no \algname{UBB-PFS} determina sub-problemas
que são partes do espaço de busca inteiro. Inspirados com esta 
estratégia, definimos um particionamento que divide o espaço de busca em 
partes iguais no algoritmo \algname{PUCS}. Este algoritmo possui 
parâmetros $p$, $l$ e algoritmo base, que são capazes de modificar o 
comportamento deste algoritmo de maneira que ele possa ser ótimo ou 
sub-ótimo. No caso ótimo, o \algname{PUCS} não teve melhor desempenho do que outros algoritmos, como
o próprio \algname{UBB-PFS}. No caso sub-ótimo notamos que, os 
parâmetros $p$ e $l$ determinam a qualidade da solução encontrada 
assim como o tempo de execução. Com um ajuste fino destes parâmetros, o
\algname{PUCS} conseguiu encontrar soluções melhores (menor custo) do 
que outras heurísticas como \algname{SFFS} e \algname{BFS}.

Finalmente, aplicamos o algoritmo \algname{PUCS} na seleção de 
características para seleção modelos de aprendizado em problemas reais
de classificação, disponíveis no UCI Machine Learning Respository. 
Utilizamos classificadores do tipo SVM linear, considerando todas as 
características e apenas as características selecionadas pelo 
\algname{PUCS}, e comparamos os resultados fazendo a validação cruzada
dos dois modelos. Os resultados mostraram que mesmo com menos 
características, o erro de validação não cresce significativamente. Isto 
é, a seleção de características nos permitiu diminuir a complexidade dos
modelos sem comprometer a qualidade dos classificadores.

\section{Trabalhos futuros}
% - percorrimento do UBB-PFS que distribua o trabalho de forma melhor
% - estudo destes novos algoritmos quanto a robustez
% - identificação de vias de sinalização celular

Listaremos agora possíveis trabalhos futuros nesta linha de pesquisa:
\begin{itemize}
    \item{Melhorias no \algname{UBB-PFS}:}  este algoritmo faz na sua primeira etapa uma busca em profundidade para criar raízes que 
determinam uma divisão do espaço de busca em sub-árvores. 
        Entretanto, este procedimento cria sub-árvores de tamanhos 
        muito diferentes, o que deve causar um grande desbalanço de
        trabalho entre linhas de processamento na paralelização. É 
        possível que em uma primeira etapa menos ingênua crie-se 
        sub-árvores que possuam tamanhos parecidos. Além disso, este 
        algoritmo, assim como o \algname{PUCS}, possui uma estrutura
        recursiva e pode ser uma heurística, se utilizarmos um algoritmo
        sub-ótimo na solução das sub-árvores, o que não foi explorado
        neste trabalho e ainda pode ser investigado.
    \item{Estudo de robustez dos novo algoritmos:} as funções de custo
        que utilizamos nas instâncias artificiais neste trabalho são
        todas decomponíveis em curvas U. Podemos refazer os mesmos 
        testes utilizando uma função com violações da propriedade da
        curva em U; por exemplo, a função da equação~\ref{cost_function:esmaeil}, que permite incluir violações da curva em U de forma controlada. Dessa forma, poderíamos estudar o que acontece com a qualidade das soluções
        geradas pelo \algname{PUCS} e \algname{UBB-PFS}. Uma observação 
        interessante, por exemplo, é que os parâmetros $p$ e $l$ do
        \algname{PUCS} devem interferir na qualidade da solução mesmo
        quando o algoritmo base é ótimo.
    \item{Seleção de características em identificação de vias de 
        sinalização celular:} na área de biologia de sistemas, chamamos
        de modelo funcional um modelo computacional capaz de simular
        fenômenos celulares. Uma das etapas na construção de um modelo
        funcional é a escolha de um conjunto de interações químicas que
        participam da via de sinalização que controla predominantemente 
        o fenômeno observado. Neste sentido, Lulu Wu apresentou 
        em sua tese de mestrado~\cite{Wu15} uma maneira de 
        sistematicamente modificar modelos funcionais ao adicionar 
        interações de um banco de dados. Entretanto sua abordagem 
        mostrou algumas limitações, e entre elas citamos a dinâmica
        incremental da estratégia de modificação, que não é capaz de remover
        interações já presentes no modelo. Desta maneira, sugerimos uma
        abordagem que enfrente esta limitação tratando a modificação
        sistemática de um modelo como um problema de seleção de
        características em que o conjunto $S$ de características é
        formado pelo banco de interações químicas e a função custo $c$ 
        é uma função que mede a qualidade do modelo que possui
        um determinado conjunto de interações (características).
\end{itemize}


\newpage
\printbibliography

\end{document}

