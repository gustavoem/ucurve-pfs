Nesta seção vamos fazer uma revisão deste trabalho, apontando e 
discutindo os resultados obtidos. Também vamos apresentar algumas opções
de linhas de pesquisa para trabalhos futuros relacionados a este.
\section{Revisão dos resultados}
% - estudamos o ubb e o pfs e com isso pudemos
%   - modificar a escolha das raízes
%     - uma escolha arbitrária parece ser melhor do que escolher uma
%       raiz que tem a maior sub-árvore.
%   - modificar estrutura de dados para escolha de raíz
%     - não foi eficiente
%   - paralelizar o PFS
%     - apesar da ramificação ser disjunta, a etapa de atualização
%       compromete o desempenho da paralelização
%   - criar um algoritmo que divide o trabalho
% - usando a ideia de dividir o trabalho, criamos o PUCS
%   - particionamento do espaço em partes iguais - melhor distribuição
%     de trabalho
%   - parâmetros que permitem controlar a granularidade do 
%     particionamento
%   - como algoritmo ótimo é mais lento que outros algoritmos
%   - como heurística acha melhores soluções
% - aplicação em seleção de modelos
%   - diminuimos a quantidade de características mantendo a qualidade
%     dos modelos de aprendizado

\section{Trabalhos futuros}
% - percorrimento do UBB-PFS que distribua o trabalho de forma melhor
% - 
