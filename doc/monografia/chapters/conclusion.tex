Nesta seção vamos fazer uma revisão deste trabalho, apontando e 
discutindo os resultados obtidos. Também vamos apresentar algumas opções
de linhas de pesquisa para trabalhos futuros relacionados a este.
\section{Revisão do trabalho}
% - estudamos o ubb e o pfs e com isso pudemos
%   - modificar a escolha das raízes
%     - uma escolha arbitrária parece ser melhor do que escolher uma
%       raiz que tem a maior sub-árvore.
%   - modificar estrutura de dados para escolha de raíz
%     - não foi eficiente
%   - paralelizar o PFS
%     - apesar da ramificação ser disjunta, a etapa de atualização
%       compromete o desempenho da paralelização
%   - criar um algoritmo que divide o trabalho
% - usando a ideia de dividir o trabalho, criamos o PUCS
%   - particionamento do espaço em partes iguais - melhor distribuição
%     de trabalho
%   - parâmetros que permitem controlar a granularidade do 
%     particionamento
%   - como algoritmo ótimo é mais lento que outros algoritmos
%   - como heurística acha melhores soluções
% - aplicação em seleção de modelos
%   - diminuimos a quantidade de características mantendo a qualidade
%     dos modelos de aprendizado

Neste trabalho, definimos o problema de seleção de características
e com o exemplo de sua aplicação em seleção de modelos definimos também
o problema U-curve, que foi o problema principal tratado aqui. Estudamos
algoritmos da literatura, investigando e sugerindo modificações a estes,
visando melhorar o seu desempenho computacional. Aplicamos a seleção de
características, feita como uma aproximação pelo problema U-curve, em
conjuntos de dados reais e verificamos como a seleção de características
pode ser benéfica na seleção de modelos de aprendizado.

Estudamos algoritmos baseados em florestas da literatura, \algname{UBB}
e \algname{PFS} e investigamos modificações ao último. Mudamos a 
estratégia de escolha de raízes para ramificação assim como a estrutura
de dados para armazenamento destas raízes no \algname{PFS}, entretanto
estas modificações não implicaram em melhoras significativas no 
desempenho do algoritmo. Sugerimos uma paralelização do algoritmo
\algname{PFS}, justificada pelo fato da etapa de ramificação poder 
acontecer de maneira independente, mas a etapa de atualização, que 
ocorre após a ramificação, mostrou-se dependente e complicada de ser
paralelizada, o que implicou em um desempenho pior do que a versão
sequencial do mesmo algoritmo. 

Ainda no escopo de modificações do \algname{UBB} e \algname{PFS}, 
propomos o \algname{UBB-PFS}, um algoritmo que utiliza o \algname{UBB} 
para decompor o espaço de busca em sub-árvores que são intervalos
do reticulado Booleano. Conseguimos reduzir a solução de cada sub-árvore
a um problema de seleção de características auxilar, em que cada um 
destes pode ser resolvido pelo algoritmo \algname{PFS} de maneira 
independente e paralela. Com a criação de sub-problemas, a paralelização 
deste algoritmo tornou-se mais simples e menos entrelaçada do que na 
outra tentativa de paralelização do \algname{PFS}. Como resultado, o 
novo algoritmo teve tempo de execução e número de chamadas da função 
custo intermediários, com menos nós computados do que o \algname{UBB} e
menor tempo de execução do que o \algname{PFS}. 

A criação de árvores feita no \algname{UBB-PFS} determina sub-problemas
que são partes do espaço de busca inteiro. Inspirados com esta 
estratégia, definimos um particionamento que divide o espaço de busca em 
partes iguais no algoritmo \algname{PUCS}. Este algoritmo possui 
parâmetros $p$, $l$ e algoritmo base, que são capazes de modificar o 
comportamento deste algoritmo de maneira que ele possa ser ótimo ou 
sub-ótimo. No caso ótimo, o \algname{PUCS} não teve melhor desempenho do que outros algoritmos, como
o próprio \algname{UBB-PFS}. No caso sub-ótimo notamos que, os 
parâmetros $p$ e $l$ determinam a qualidade da solução encontrada 
assim como o tempo de execução. Com um ajuste fino destes parâmetros, o
\algname{PUCS} conseguiu encontrar soluções melhores (menor custo) do 
que outras heurísticas como \algname{SFFS} e \algname{BFS}.

Finalmente, aplicamos o algoritmo \algname{PUCS} na seleção de 
características para seleção modelos de aprendizado em problemas reais
de classificação, disponíveis no UCI Machine Learning Respository. 
Utilizamos classificadores do tipo SVM linear, considerando todas as 
características e apenas as características selecionadas pelo 
\algname{PUCS}, e comparamos os resultados fazendo a validação cruzada
dos dois modelos. Os resultados mostraram que mesmo com menos 
características, o erro de validação não cresce significativamente. Isto 
é, a seleção de características nos permitiu diminuir a complexidade dos
modelos sem comprometer a qualidade dos classificadores.

\section{Trabalhos futuros}
% - percorrimento do UBB-PFS que distribua o trabalho de forma melhor
% - estudo destes novos algoritmos quanto a robustez
% - identificação de vias de sinalização celular

Listaremos agora possíveis trabalhos futuros nesta linha de pesquisa:
\begin{itemize}
    \item{Melhorias no \algname{UBB-PFS}:}  este algoritmo faz na sua primeira etapa uma busca em profundidade para criar raízes que 
determinam uma divisão do espaço de busca em sub-árvores. 
        Entretanto, este procedimento cria sub-árvores de tamanhos 
        muito diferentes, o que deve causar um grande desbalanço de
        trabalho entre linhas de processamento na paralelização. É 
        possível que em uma primeira etapa menos ingênua crie-se 
        sub-árvores que possuam tamanhos parecidos. Além disso, este 
        algoritmo, assim como o \algname{PUCS}, possui uma estrutura
        recursiva e pode ser uma heurística, se utilizarmos um algoritmo
        sub-ótimo na solução das sub-árvores, o que não foi explorado
        neste trabalho e ainda pode ser investigado.
    \item{Estudo de robustez dos novo algoritmos:} as funções de custo
        que utilizamos nas instâncias artificiais neste trabalho são
        todas decomponíveis em curvas U. Podemos refazer os mesmos 
        testes utilizando uma função com violações da propriedade da
        curva em U e estudar o que acontece com a qualidade das soluções
        geradas pelo \algname{PUCS} e \algname{UBB-PFS}. Uma observação 
        interessante, por exemplo, é que os parâmetros $p$ e $l$ do
        \algname{PUCS} devem interferir na qualidade da solução mesmo
        quando o algoritmo base é ótimo.
    \item{Seleção de características em identificação de vias de 
        sinalização celular:} na área de biologia de sistemas, chamamos
        de modelo funcional um modelo computacional capaz de simular
        fenômenos celulares. Uma das etapas na construção de um modelo
        funcional é a escolha de um conjunto de interações químicas que
        participam da via de sinalização que controla predominantemente 
        o fenômeno observado. Neste sentido, Lulu Wu apresentou 
        em sua tese de mestrado~\cite{Wu15} uma maneira de 
        sistematicamente modificar modelos funcionais ao adicionar 
        interações de um banco de dados. Entretanto sua abordagem 
        mostrou algumas limitações, e entre elas citamos a dinâmica
        incremental de sua abordagem, que não é capaz de remover
        interações já presentes no modelo. Desta maneira, sugerimos uma
        abordagem que enfrente esta limitação tratando a modificação
        sistemática de um modelo como um problema de seleção de
        características em que o conjunto $S$ de características é
        formado pelo banco de interações químicas e a função custo $c$ 
        é uma função que mede a qualidade do modelo que possui
        um determinado conjunto de interações (características).
\end{itemize}
