% - machine learning e seleção de características
%    problema: falta de amostras
%    solução: simplificar o modelo de aprendizado -> seleção de 
%             características    
% - aplicações: w-operadores, construção de modelos funcionais
% - problema de otimização
% - funções de custo
% - algoritmos de seleção de características
% - trabalhos antigos

A seleção de características pode ser utilizada como um auxílio na
construção de um modelo de aprendizado de máquina. Essa técnica consiste
em, dado o conjunto de características observadas nas amostras, escolher
um subconjunto que seja ótimo de acordo com alguma métrica. Devemos 
considerar a etapa de seleção de características na construção de um 
modelo de aprendizado quando a quantidade de características é muito
grande, o que pode fazer o modelo ser muito caro computacionalmente; ou
quando a quantidade de amostras é pequena comparada a complexidade do 
modelo original, em outras palavras, quando ocorre sobreajuste (do 
inglês, \foreignword{overfitting}).
