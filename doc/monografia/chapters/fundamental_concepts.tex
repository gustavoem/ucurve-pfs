% Contextualização e conceitos fundamentais
% .1 O problema de seleção de características
% .2 Funções de custo para esse problema
% .3 Redução para o problema de curvas em U

\section{O problema de seleção de características}
A seleção de características é um problema de otimização combinatória 
em que procuramos o melhor subconjunto de um conjunto de características
$S$. O espaço de busca desse problema é o conjunto potência de $S$, 
$\powerset (S)$, que é uma coleção de todos os subconjuntos possíveis de
 $S$. A função de custo desse problema é uma função $c : \powerset (S) 
\to \fieldR_{+}$.

\begin{mydefinition}[Problema de seleção de características] Seja $S$
um conjunto finito não vazio de características e $c$ uma função de 
custo. Encontrar $X \in \powerset (S)$ tal que $c (X) \leq c (Y)$,
$\forall Y \in \powerset (S)$.
\end{mydefinition}

O espaço de busca do problema de seleção de características possui uma
relação de ordem parcial definida pela relação $\subseteq$, portanto
este conjunto é {\bf parcialmente ordenado (poset)}.

\begin{mydefinition}
Uma {\bf \em cadeia} do reticulado booleano é uma sequência $X_1$, 
$X_2$, ..., $X_l$ tal que $X_1 \subset X_2 \subset \dots \subset X_l$.
\end{mydefinition}

\begin{mydefinition}
Uma cadeia é dita {\bf \em maximal} se não existe outra cadeia no 
reticulado que contenha propriamente esta cadeia.
\end{mydefinition}

\begin{mydefinition}
Uma função de custo $c$ é dita decomponível em curvas u se para toda 
cadeia maximal $X_1, ..., X_l$, $c(X_j) \leq min \{c (X_i),
c (X_k)\}$ sempre que $X_i \subset X_j \subset X_k$, $i, j, k \in 
\{1, ..., l\}$.
\end{mydefinition}

\section{Funções de custo para o problema de seleção de características}
A função de custo utilizada na solução do problema de seleção de 
características deve, de alguma forma, refletir a qualidade do conjunto 
avaliado. Por isso, diferentes aplicações de seleção de características
podem ter diferentes funções de custo. No contexto de aprendizado de 
máquina, uma possível função de custo é a entropia média condicional
(MCE), que já foi utilizada por exemplo na construção de W-operadores
~\cite{MJCJB06}.

\begin{mydefinition}
Dado um conjunto de treinamento com $m$ amostras 

\end{mydefinition}
