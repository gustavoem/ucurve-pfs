O algoritmo \algname{Parallel U-Curve Search} (PUCS) foi desenvolvido 
para resolver o problema U-Curve particionando o espaço de busca em 
partes que podem ser resolvidas independentemente e de forma paralela. 
Além disso, a dinâmica desse algoritmo depende de parâmetros que 
determinam o tempo de execução e qualidade da solução obtida, permitindo
ao usuário adequar o algoritmo aos recursos computacionais disponíveis. 

\section{Princípios do algoritmo}
 
\subsection{Partição do espaço de busca}
Seja $S$ o conjunto de características do problema em questão. O 
primeiro passo do particionamento é escolher arbitrariamente $S'$ um 
subconjunto de $S$; de maneira complementar, definimos 
$\overline{S'} = S \setminus S'$. Agora, sejam $X, Y \in \powerset (S)$
e $\sim$ a relação:

\begin{equation*}
    X \sim Y \iff (X \cap S') = (Y \cap S')
\end{equation*}
Esta relação é de equivalência, pois nela valem:

\begin{itemize}
    \item{reflexividade}
        \begin{align*} 
            X \sim X \text{, pois }
            (X \cap S') = (X \cap S')
        \end{align*} 
    \item{simetria}
        \begin{align*}
            X \sim Y  & \iff \\
            (X \cap S') = (Y \cap S') & \iff \\
            (Y \cap S') = (X \cap S') & \iff \\
            Y \sim X 
        \end{align*}
    \item{transitividade,}
        \begin{align*}
            X \sim Y, Y \sim Z & \Rightarrow \\
            (X \cap S') = (Y \cap S') = (Z \cap S') & \Rightarrow \\
            (X \cap S') = (Z \cap S') & \Rightarrow \\
            X \sim Z
        \end{align*}
\end{itemize}
Portanto, o conjunto das classes de equivalência definidas por $\sim$ é
uma partição do espaço de busca original. Tome como exemplo o conjunto
$S = \{a, b, c\}$; se $S' = {a}$, então existem duas classes de 
equivalência no particionamento do espaço de busca que definimos, 
formados pelos conjuntos $\{\emptyset, b, c, bc\}$ e $\{a, ab, ac, 
abc\}$.

Pela definição da relação $\sim$ temos que a presença de cada 
característica de $S'$ em uma dada parte do reticulado não muda, isto é,
ou ela está presente em todos subconjuntos da parte ou não está presente
em nenhum, portanto, dizemos que estas variáveis são {\bf fixas}. De 
modo análogo, as variáveis de $\overline{S'}$ são {\bf livres}. Tanto 
variáveis fixas quanto livres podem definir reticulados booleanos junto 
a relação de ordem parcial $\subseteq$.

O conjunto $\powerset (S')$ induz um reticulado booleano em que cada
elemento representa uma classe de equivalência do espaço de soluções
do problema original, chamamos este de {\bf reticulado externo}. Para 
cada classe de equivalência (nó do reticulado externo), o conjunto 
$\powerset (\overline {S'})$ induz um outro reticulado booleano ({\bf
reticulado interno}) em que cada elemento representa um subconjunto de
problema original. Seja $A \in \powerset (S')$ um elemento do reticulado 
externo, então cada $B \in \powerset (\overline{S'})$ do reticulado 
interno em $A$ representa o conjunto $X = B \cup A$ do espaço de busca
do problema original.

Os reticulados internos e externo elucidam a estrutura recursiva do 
problema de seleção de características e sugerem que podemos construir 
uma solução ao problema original a partir de soluções de outros 
problemas, sobre os reticulados externo e internos, abordagem conhecida
em computação como divisão e conquista. Seja $\langle S, c \rangle$ uma 
instância do problema de seleção de características, $S'$ o conjunto de 
variáveis fixas, $\overline{S'}$ o conjunto de variáveis livres, e 
$A \in \powerset (S')$ um subconjunto que é nó do reticulado externo, 
então podemos definir um outro problema de seleção de características 
$\langle \overline{S'}, c_{A} \rangle$ em que 
\begin{align*}
    c_{A} (X) = c (X \cup A).
\end{align*}
Resolver a instância $\langle \overline{S'}, c_{A} \rangle$ é 
essencialmente achar o mínimo do problema inicial restrito a classe de
equivalência de $A$, dizemos também que estamos resolvendo a parte $A$. 
Se soubermos em qual classe o mínimo global reside, podemos resolver 
apenas tal parte e garantir que a solução encontrada é a solução do 
problema original.

O algoritmo PUCS percorre o reticulado externo recolhendo partes 
candidatas a conter o mínimo e resolve cada uma destas, escolhendo o 
mínimo das soluções destas partes como o mínimo global do problema.

\subsection{Percorrimento do reticulado externo}
