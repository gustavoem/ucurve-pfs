\documentclass{standalone}
\usepackage{tikz}
\usepackage{tikz-qtree,tikz-qtree-compat}
\usepackage{my_colors}
\usetikzlibrary{calc}

\begin{document}


  \tikzset{
    leftNode/.style={circle,minimum width=.5ex, line width=.5pt, fill=my_blue,draw},
    rightNode/.style={circle,minimum width=.5ex, line width=.5pt, fill=my_red,thick,draw},
    rightNodeInLine/.style={solid,circle,minimum width=.7ex, fill=black,thick,draw=white},
    leftNodeInLine/.style={solid,circle,minimum width=.7ex, fill=none,thick,draw},
  }
  \begin{tikzpicture}[
        important line/.style={thick}, dashed line/.style={dashed, thin},
        every node/.style={color=black},
    ]
    %\draw[dashed line, yshift=.7cm]
       %(.2,.2) coordinate (sls) -- (2.5,2.5) coordinate (sle)
       %node[solid,circle,minimum width=2.8ex,fill=none,thick,draw] (name) at (2,2){}
       %node[leftNodeInLine] (name) at (2,2){}
       %node[solid,circle,minimum width=2.8ex,fill=none,thick,draw] (name) at (1.5,1.5){}
       %node[leftNodeInLine] (name) at (1.5,1.5){}
       %node [above right] {$w\cdot x + b > 1$};

    \draw[important line]
       (0,0) coordinate (lines) -- (4,4) coordinate (linee)
       node [above right] {hiperplano};


    \foreach \Point in {(1,4), (2,3), (1.3,2.1), (2,3), (1,2.9), (-.2, 2), (.3, 3.5),
                        (3.2,3.7)}{
      \draw \Point node[leftNode]{};
    }

    \foreach \Point in {(3.7,1.2), (3,2), (3,0), (1.5,1), (2.5,1), (4, 3.2),
                        (3.8, 2.5)}{
      \draw \Point node[rightNode]{};
    }
  \end{tikzpicture}
\end{document}
