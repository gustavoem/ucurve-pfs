\documentclass[12pt]{article}
\usepackage[portuguese]{babel}
\usepackage[utf8]{inputenc}
\usepackage[usenames,dvipsnames]{color}
\usepackage{setspace}
\usepackage{amsmath}
\usepackage{amsfonts}
\usepackage{amssymb}
\usepackage{mathtools}
\usepackage[top=3cm, bottom=2cm, left=3cm, right=2cm]{geometry}
\usepackage{tikz}
\usepackage{indentfirst}
\usepackage{textcomp}
\title{Relatorio IC}

% packages added by Marcelo
%
\usepackage{lscape}    % for landscape pages
\usepackage{hyperref}  % to allow hyperlinks
\usepackage{booktabs}  % nicer table borders
\usepackage{subfigure} % add subfigures

\newcommand{\foreignword}[1]{\textit{#1}}
\newcommand{\toolname}[1]{\textit{#1}}
%\newcommand{\fieldR}{\mathbb{R}}
%\newcommand{\powerset}{\mathcal{P}}
%\newcommand{\probability}{\mathbb{P}}
%\newcommand{\expectation}{\mathbb{E}}
\newcommand{\algname}[1]{\texttt{#1}}
%\newcommand{\langname}[1]{\texttt{#1}}
%\newcommand{\varname}[1]{\texttt{#1}}
%\newcommand{\floor}[1]{\lfloor #1 \rfloor}
%\newcommand{\ceil}[1]{\lceil #1 \rceil}
%\newcommand{\mathsc}[1]{{\normalfont\textsc{#1}}}
%\newcommand{\forest}{\mathcal{F}}
%\newcommand{\pfsnode}[1]{\mathbf #1}
%\newcommand{\species}[1]{\textit{#1}}
%\newcommand{\gender}[1]{\textit{#1}}

\graphicspath{{./figures/}} 

\setstretch{1.5}

\begin{document}

% FAPESP demands the usage of double spacing
%
\doublespacing

\begin{titlepage}
    \vfill 
    \begin{center}
        {\Large Relatório Científico Final -- Iniciação Científica\\
         \bigskip
         Processo FAPESP 2016/25959-7
        }
        
        \bigskip
        \bigskip
    
        {\LARGE Projeto de algoritmos baseados em florestas de posets 
                para o problema de otimização U-curve}

        \bigskip
        \bigskip
        {\Large {\bf Beneficiário:} \href{mailto:gustavo.estrela.matos@usp.br}{Gustavo Estrela de Matos}\\ 
        
        {\bf Responsável:} \href{mailto:marcelo.reis@butantan.gov.br}{Marcelo da Silva Reis}\\

        \bigskip
        \bigskip
        \bigskip
        \bigskip
        \bigskip
        \bigskip
        \bigskip
Relatório referente aos trabalhos desenvolvidos entre 1 de maio e 31 de dezembro de 2017

        \bigskip
        \bigskip
        \bigskip
        \bigskip
        \bigskip
        \bigskip
        \bigskip

Laboratório Especial de Toxinologia Aplicada, Instituto Butantan\\
        \bigskip
        São Paulo, \today\\
        }

        \bigskip
        \bigskip

       

\end{center}
\end{titlepage}


\tableofcontents

\pagebreak



\section{Resumo do Projeto Proposto} \label{sec:resumo} % até 2 páginas
O problema U-curve é uma formulação de um problema de otimização que 
pode ser utilizado na etapa de seleção de características em 
Aprendizado de Máquina, com aplicações em desenho de modelos 
computacionais de sistemas biológicos. Não obstante, soluções propostas 
até o presente momento para atacar esse problema têm limitações do 
ponto de vista de consumo de tempo computacional e/ou de memória, o que 
implica na necessidade do desenvolvimento de novos algoritmos. Nesse 
sentido, em 2012 foi proposto o algoritmo 
\algname{Poset\--Forest\--Search} (\algname{PFS}), 
que organiza o espaço de busca em florestas de posets. Esse algoritmo 
foi implementado e testado, com resultados promissores; todavia, novos 
melhoramentos são necessários para que o \algname{PFS} se torne uma 
alternativa competitiva para resolver o problema U-curve. Neste projeto 
propomos modificações ao \algname{PFS} na escolha de caminhos de 
percorrimento da floresta de busca, e na estrutura de dados utilizada 
para armazenar este grafo, com o uso de diagramas de decisão binária 
reduzidos e ordenados (OBDDs); também propomos a criação de uma versão 
paralela e escalável do algoritmo \algname{PFS}. Além disso, propomos a 
criação de um algoritmo baseado no \algname{PFS} que tenha 
características de um algoritmo de aproximação, no qual o critério de 
aproximação da solução ótima se baseie no teorema da navalha de Ockham. 
Os algoritmos desenvolvidos serão implementados no arcabouço 
\toolname{featsel} e testados com instâncias artificiais e também reais,
com conjuntos de dados de aprendizado de máquina retirados do University 
of California Irvine (UCI) Machine Learning Repository.

 
\section{Atividades Realizadas}
\subsection{Estudo de algoritmos baseados em florestas}
O algoritmo \algname{Poset\--Forest\--Search} (\algname{PFS}) é um 
algoritmo ótimo para resolver o problema de otimização U-Curve e serviu 
de base para a criação da maioria dos algoritmos elaborados neste 
trabalho. O \algname{PFS} é uma generalização de um outro algoritmo mais
simples, o \algname{U-curve-Branch-and-Bound} (\algname{UBB}), que é
um algoritmo \foreignword{branch-and-bound} ótimo que decompõe o espaço
de busca em uma árvore, e acha o mínimo global do problema fazendo 
ramificações e podas nesta árvore.

A árvore de busca do \algname{UBB} permite que a procura pelo mínimo 
ocorra de maneira parecida com uma busca em profundidade, que percorre
cadeias do reticulado Booleano de subconjuntos menores para maiores. 
Sempre que o custo de um subconjunto $X_i$ aumenta em comparação ao 
anterior $X_j$ no percorrimento, a hipótese de que a função de custo é 
decomponível em curvas em U garante que a subárvore que começa em $X_i$
pode ser removida do espaço de busca; chamamos este procedimento de
poda. O algoritmo \algname{UBB} tem, entretanto, uma limitação, pois
quando a função de custo do problema é monótona não-crescente, a 
condição de poda nunca é verdadeira e o espaço de busca inteiro é 
visitado, o que compromete a escalabilidade do algoritmo.

O \algname{PFS} enfrenta esta limitação ao fazer percorrimentos de
cadeias do espaço de busca em duas direções, de conjuntos menores para
maiores (como faz o \algname{UBB}) e também o contrário. Para fazer 
isso, este algoritmo decompõe o espaço de busca em duas árvores, uma 
para cada direção de percorrimento. Com a criação de duas estruturas 
para representar o mesmo espaço de busca, torna-se necessário a 
atualização de uma estrutura sempre que a outra sofrer mudanças, e isto
implica na utilização de florestas ao invés de árvores para representar
o espaço de busca no \algname{PFS}. Resumidamente, uma iteração do 
deste algoritmo deve escolher uma direção de percorrimento; fazer o 
percorrimento com poda (de maneira similar ao \algname{UBB}); e, 
finalmente, atualizar a floresta dual a que foi percorrida.

\subsection{Modificações do PFS na escolha de raízes}
\subsection{Mudificação do PFS no armazenamento de raízes}
\subsection{Paralelização do PFS}
\subsection{Elaboração do UBB-PFS}
\subsection{Elaboração de um algoritmo de aproximação}
\subsection{Testes com instâncias reais do problema de seleção de 
            características}
\section{Avaliação e disseminação de resultados}
\section{Conclusão}
\pagebreak


\begin{thebibliography}{9} \label{sec:referencias}

\addcontentsline{toc}{section}{Referências}

\bibitem{msreis thesis}
Reis, Marcelo S. ``Minimization of decomposable in U-shaped curves functions defined on poset chains–algorithms and applications." PhD thesis, Institute of Mathematics and Statistics, University of São Paulo, Brazil, (2012).

\bibitem{ucs paper}
Reis, Marcelo S., Carlos E. Ferreira, and Junior Barrera. ``The U-curve optimization problem: improvements on the original algorithm and time complexity analysis." arXiv preprint arXiv:1407.6067, (2014). 


\bibitem{bryant}
Bryant, Randal E. ``Graph-based algorithms for boolean function manipulation." IEEE Transactions on Computers, 100.8 (1986): 677-691. 


\end{thebibliography}

\end{document}


