\documentclass[12pt]{article}
\usepackage[portuguese, brazilian, english]{babel}
\usepackage[utf8]{inputenc}
\usepackage[usenames,dvipsnames]{color}
\usepackage{setspace}
\usepackage{amsmath}
\usepackage{amsfonts}
\usepackage{amssymb}
\usepackage{mathtools}
\usepackage[top=3cm, bottom=2cm, left=3cm, right=2cm]{geometry}
\usepackage{tikz}
\usepackage{textcomp}
\usepackage{lscape}    % for landscape pages
\usepackage{hyperref}  % to allow hyperlinks
\usepackage{booktabs}  % nicer table borders
\usepackage{subfigure} % add subfigures

\title{Projeto IC PFS}

% Figures directory
\graphicspath{{./figures/}} 

\definecolor{myblue}{RGB}{80,80,160}
\definecolor{mygreen}{RGB}{80,160,80}
\setstretch{1.5}

\begin{document}

% FAPESP demands the usage of double spacing
%
\doublespacing

\selectlanguage{brazilian}
\thispagestyle{empty} 
\begin{flushright}
    {\LARGE Projeto de algoritmos baseados em florestas de posets\\
        \bigskip 
        para o problema de otimização U-curve}

  \bigskip
  \bigskip
        
  {\large {\bf Beneficiário:} \href{mailto:gustavo.estrela.matos@usp.br}{Gustavo Estrela de Matos}\\ 
  {\bf Pesquisador Responsável:} \href{mailto:marcelo.reis@butantan.gov.br}{Marcelo da Silva Reis}\\
  \bigskip
Centro de Toxinas, Imuno-resposta e Sinalização Celular (CeTICS)\\
Laboratório Especial de Ciclo Celular (LECC)\\
  Instituto Butantan, São Paulo, \today.\\
  }

  \bigskip
  \bigskip
\end{flushright}
\begin{abstract}
% Resumo
O problema U-curve é uma formulação de um problema de otimização que pode ser utilizado na etapa de seleção de características em Aprendizado de Máquina, com aplicações em desenho de modelos computacionais de sistemas biológicos. Não obstante, soluções propostas até o presente momento para atacar esse problema têm limitações do ponto de vista de consumo de tempo computacional e/ou de memória, o que implica na necessidade do desenvolvimento de novos algoritmos. Nesse sentido, em 2012 foi proposto o algoritmo Poset-Forest-Search ({\tt PFS}), que organiza o espaço de busca em florestas de posets. Esse algoritmo foi implementado e testado, com resultados promissores; todavia, novos melhoramentos são necessários para que o {\tt PFS} se torne uma alternativa competitiva para resolver o problema U-curve. Neste projeto propomos a construção de uma versão paralelizada e escalável do algoritmo {\tt PFS}, utilizando diagramas de decisão binária reduzidos e ordenados. Além disso, propomos adaptar o {\tt PFS} como um algoritmo de aproximação, no qual o critério de aproximação da solução ótima faça uso do teorema da navalha de Ockham. Os algoritmos desenvolvidos serão implementados e testados em instâncias artificiais e também em conjuntos de dados próprios para experimentos comparativos entre diferentes algoritmos de seleção de características.
\end{abstract}

\newpage
\thispagestyle{empty} 
\selectlanguage{english}
\begin{flushright}
    {\LARGE Design of poset forest-based algorithms for the\\
        \bigskip 
        U-curve optimization problem}

  \bigskip
  \bigskip
        
  {\large {\bf Student:} \href{mailto:gustavo.estrela.matos@usp.br}{Gustavo Estrela de Matos}\\ 
  {\bf Supervisor:} \href{mailto:marcelo.reis@butantan.gov.br}{Marcelo da Silva Reis}\\
  \bigskip
Center of Toxins, Immune-response and Cell Signaling (CeTICS)\\
Laboratório Especial de Toxinologia Aplicada (LETA)\\
  Instituto Butantan, São Paulo, \today.\\
  }
  \bigskip
  \bigskip
\end{flushright}
\begin{abstract}
% Abstract
The U-curve problem is a formulation of an optimization problem that can be used in the feature selection step of Machine Learning, with applications in the designing of computational models of biological systems. Nevertheless, the solutions so far proposed to tackle this problem have limitations from both required computational time and space points of view, which implies in the need for development of new algorithms. To this end, it was introduced in 2012 the Poset-Forest-Search ({\tt PFS}) algorithm, which organizes the search space in forests of posets. This algorithm was implemented and tested, with promising results; however, new improvements are required until {\tt PFS} becomes a competitive alternative to tackle the U-curve problem. In this project, we propose the design of a parallelized, scalable version of the {\tt PFS} algorithm, using reduced ordered binary decision diagrams. Moreover, we propose to adapt {\tt PFS} as an approximation algorithm, in which the approximation criterion to the optimal solution makes use of the Ockham's razor theorem. The developed algorithms will be implemented and tested on artificial instances and also on collection of datasets that are suitable for benchmarking of feature selection algorithms.
\end{abstract}

    
\newpage
\selectlanguage{brazilian}
\tableofcontents
\newpage

\section{Introdução}
% Apresentação do problema U-curve (na linha do projeto da primeira IC,
% só que procurando ser mais sucinto);
% Recapitulação da IC anterior (melhoramentos do algoritmo UCS com 
% ROBDDs), com particular destaque para a limitação dos melhoramentos 
% obtidos.
% 
\subsection{O problema U-curve}
O problema de seleção de característica consiste em, dado um conjunto $S$
de características, escolher um subconjunto de $S$ que seja
ótimo de acordo com
alguma métrica. A solução desse problema tem aplicações em
Aprendizado de Máquina e Reconhecimento de Padrões, com aplicações práticas
em identificação de sistemas biológicos -- por exemplo, na estimação de redes
gênicas regulatórias. Formalmente, podemos 
definir o problema de seleção de características como um problema de 
busca, no qual procuramos por um subconjunto $X \in \mathcal{P}(S)$ que 
minimiza uma função de custo $c : \mathcal{P}(S) \to \mathbb{R^+}$.
O espaço de busca do problema de seleção de características pode ser
estruturado como um reticulado Booleano ($\mathcal{P}(S)$, $\subseteq$), no qual
cada elemento é um subconjunto de características. Em funções custo $c$ que dependem
da estimação de uma distribuição conjunta a partir de um número limitado de
amostras, é comum que o custo das cadeias do reticulado, quando restritas a $c$,
descrevam curvas em U (figura~\ref{fig:U-curve}); esse
comportamento pode ser intuitivamente explicado se considerarmos que o
erro de estimação diminui ao adicionarmos novas características, até um ponto
em que a limitação de amostras faz com que acrescentar características
leve a um aumento no erro de estimação.

\begin{figure}[h]
\centering 
\begin{tabular}{c c}
    \subfigure[] {
        \scalebox{0.75}{
        \includegraphics
        [trim=2cm 17.5cm 12.2cm 2cm,clip=true]{Boolean_lattice_3_A.pdf}}
        \label{fig:U-curve-example:A}
    }  
    &
    \subfigure[] {
        \scalebox{0.30}{
        \includegraphics[clip=true]{exemplo1.pdf}} 
        \label{fig:U-curve-example:B} 
    }
\end{tabular}
\caption{exemplo de instância do problema U-curve. 
Figura~\ref{fig:U-curve-example:A}: o diagrama de Hasse de um reticulado
Booleano de grau $3$ -- as cadeias do reticulado, cujos custos de seus
elementos são definidos através dos números ao lado dos elementos, descrevem
curvas em U; a cadeia maximal $\{ \emptyset, c, bc, abc \}$ está
destacada em negrito. O elemento $bc$, destacado em verde, é mínimo na
cadeia (e também no reticulado Booleano). 
Figura~\ref{fig:U-curve-example:B}: o gráfico dos custos em função dos
elementos da cadeia maximal destacada em negrito, mostrando sua curva em
U. Figuras extraídas de Reis~\cite{msreis thesis}.} 
    \label{fig:U-curve} 
\end{figure}



O problema U-curve é um caso particular do problema de seleção de
características no qual todas as cadeias do espaço de busca descrevem
curvas em U quando avaliadas pela função de custo (figura~\ref{fig:U-curve}).
Existem algoritmos
ótimos para solução do problema U-curve, tais como o U-curve-Branch-and-Bound
({\tt UBB}) e o Poset-Forest-Search ({\tt PFS})~\cite{msreis thesis}; os
princípios de funcionamento deste último serão apresentados a seguir.


\subsection{O algoritmo Poset-Forest-Search ({\tt PFS})}
O algoritmo {\tt PFS} é um algoritmo ótimo do tipo 
{\em branch-and-bound}. A ideia básica é generalizar o algoritmo {\tt UBB},
que é um {\em branch and bound} unidirecional tanto em termos de percorrimento do espaço de busca quanto de podas, em um algoritmo bidirecional em ambos os procedimentos. Podas bidirecionais em um reticulado Booleano completo resultam em uma coleção de posets, de onde vem o nome do algoritmo.

Antes da inicialização do algoritmo, o espaço de 
busca é representado em duas árvores $T$ e $T'$ tais que $T'$ é uma árvore complementar a 
$T$, ou seja, para qualquer aresta $\{ X, Y \}$ na
árvore $T$ existe uma aresta $\{ X^c, Y^c \}$ na árvore $T'$ 
(figuras~\ref{fig:PFS_tree:A} e \ref{fig:PFS_tree:B}).


\begin{figure}[h]
  \centering 
  \begin{tabular}{c c}
    \subfigure[] {\scalebox{0.45}{
     \includegraphics[trim=3cm 8.5cm 1.2cm 2cm, clip=true]{PFS_tree_A.pdf}}
     \label{fig:PFS_tree:A} }
    &
    \subfigure[] {\scalebox{0.45}{
    \includegraphics[trim=3cm 8.5cm 1.2cm 2cm, clip=true]{PFS_tree_B.pdf}}
    \label{fig:PFS_tree:B} }
  \end{tabular}
  \caption{exemplo de enumerações de um reticulado Booleano, no qual a
árvore $T$ (figura~\ref{fig:PFS_tree:A}) é complementar à árvore $T'$
(figura~\ref{fig:PFS_tree:B}). Figuras extraídas de Reis~\cite{msreis thesis}.}
  \label{fig:PFS_tree} 
\end{figure}

O algoritmo inicia a execução escolhendo uma das duas árvores; sem perda de generalidade, digamos que $T$ foi escolhida. A partir
do conjunto vazio, uma cadeia é percorrida até que o custo de um elemento seja maior que o do elemento $X$ que o precedeu na cadeia (figura~\ref{fig:PFS_pruning:A}); nesse caso, são eliminados do espaço de busca todos os elementos que se ramificam na árvore $T$ a partir de $X$, assim como as arestas percorridas desde o conjunto vazio. Esse procedimento resulta em uma floresta $\mathcal{A}$ composta por três árvores (figura~\ref{fig:PFS_pruning:A}). Ao removermos de $T'$ os mesmos elementos que foram removidos de $T$, o grafo induzido pelas arestas remanescentes resulta em uma floresta $\mathcal{B}$ também composta por três árvores (figura~\ref{fig:PFS_pruning:B}).

O algoritmo {\tt PFS} inicia então uma nova iteração, escolhendo uma das florestas e em seguida repetindo o procedimento descrito acima para uma das árvores que a compõem. Se uma árvore contém apenas um único elemento $X$ (i.e., uma árvore trivial, composta apenas por uma raiz), então $X$ é removido da floresta escolhida, e a outra floresta é atualizada eliminando-se $X$ e as arestas com uma das pontas ligada a esse elemento. O algoritmo {\tt PFS} itera até que todo o espaço de busca seja esgotado.


\begin{figure}[h]
  \centering 
  \begin{tabular}{c c}
    \subfigure[] {\scalebox{0.45}{
    \includegraphics[trim=3cm 8.5cm 1.2cm 2cm, clip=true]{PFS_dynamics_D.pdf}}
    \label{fig:PFS_pruning:A}  }
    &
    \subfigure[] {\scalebox{0.45}{
    \includegraphics[trim=3cm 8.5cm 1.2cm 2cm, clip=true]{PFS_dynamics_G.pdf}}
    \label{fig:PFS_pruning:B}  }
  \end{tabular}
  \caption{exemplo de um percorrimento e poda realizados na árvore $T$, no qual é eliminado 
do espaço de busca o intervalo $[11100, 11111]$ e produzida uma floresta $\mathcal{A}$ contendo três árvores, cujas raízes estão destacadas em amarelo
(figura~\ref{fig:PFS_pruning:A}). A eliminação desse intervalo em $T'$ produz
uma floresta $\mathcal{B}$ também com três árvores (figura \ref{fig:PFS_pruning:B}). Figuras extraídas
de Reis~\cite{msreis thesis}.}
  \label{fig:PFS_pruning} 
\end{figure}

Uma versão preliminar do {\tt PFS} já foi implementada e mostrou resultados promissores, principalmente
quando analisado o número de elementos do reticulado que são visitados~\cite{msreis thesis}. Apesar
disso, o algoritmo apresenta propriedades interessantes que foram pouco exploradas nesse trabalho original; por exemplo, o gerenciamento do espaço de busca através de florestas poderia ser aproveitado para o desenvolvimento de algoritmos paralelizados ótimos para o problema U-curve, com maior escalabilidade em relação a variantes sequenciais. Ademais, seria interessante testar a aplicabilidade desse algoritmo em um conjunto abrangente de instâncias de interesse prático.


\section{Objetivos}
Neste trabalho, propomos atingir três objetivos:

\begin{enumerate}
\item Implementação de uma versão do algoritmo sequencial do algoritmo {\tt PFS} utilizando diagramas de decisão binária reduzidos e ordenados ({\em Reduced Ordered Binary Decision Diagrams} -- ROBDDs) para representar as listas de raízes da floresta que representa o espaço de busca no algoritmo {\tt PFS}. ROBDD é uma estrutura de dados eficiente para gerenciar coleções de elementos de um reticulado Booleano~\cite{bryant}.

\item Desenho de uma versão escalável do {\tt PFS}. Para este fim, paralelizaremos o percorrimento das
árvores nas florestas que representam o espaço de busca, com o programa
principal gerenciando a escolha das raízes (i.e., início de um 
percorrimento), guardando o mínimo corrente e centralizando a
atualização das podas.

\item Desenvolvimento de uma versão do {\tt PFS} que funcione como algoritmo 
de aproximação para o problema U-curve, utilizando como critério de
aproximação da solução ótima o teorema da navalha de Ockham: dado um espaço de hipóteses $H$ (i.e., espaço de busca), o número mínimo 
de amostras necessário para se obter uma solução que erra no máximo 
$\epsilon$ com $1 - \delta$ de probabilidade é expresso por:
\begin{equation}
\displaystyle  m(\delta,\epsilon) \ge 
    \frac{1}{\epsilon} log (\frac{|H|}{\delta}). \label{eq:Ockham}
\end{equation}
Este resultado da equação~\ref{eq:Ockham}, que é bem conhecido em teoria de Aprendizado de Máquina~\cite{kearns}, será aproveitado para o uso das variantes do algoritmo {\tt PFS} para resolver de forma aproximada instâncias de tamanho proibitivo mesmo para versões escaláveis do algoritmo ótimo.
\end{enumerate}

Para todos os algoritmos propostos, faremos implementação e testes empíricos, para isso utilizando tanto instâncias artificiais quanto conjuntos de dados próprios para {\em benchmarking} de algoritmos de seleção de características.


\section{Plano de trabalho}
A pesquisa se iniciará com o estudo do algoritmo {\tt PFS}. Uma implementação desse algoritmo atualmente encontra-se disponível para o  
arcabouço \href{https://github.com/msreis/featsel}{featsel}, e será 
usado como base para os futuros algoritmos desse projeto. O próximo 
passo envolverá a construção de uma modificação do algoritmo {\tt PFS}
que utiliza a estrutura de dados ROBDD para guardar as raízes das 
florestas que representam o espaço de busca do problema.

Estudaremos em seguida a biblioteca \href{http://www.openmp.org/}{OpenMP}, escolhida para paralelização
do algoritmo através do uso de memória compartilhada. Após concluídos os estudos dessa biblioteca, devemos analisar a
dinâmica do algoritmo {\tt PFS} e reescrevê-la com as necessárias 
modificações para que partes do algoritmo possam ser executadas em
paralelo. A princípio, a paralelização do algoritmo se dará no 
percorrimento de caminhos nas florestas que compõem o espaço de busca, que será 
atualizado em uma área da memória comum a todas as linhas de execução.
Ao final dessa etapa, teremos um novo algoritmo, que será testado
e comparado com outros algoritmos do arcabouço featsel, tais como o {\tt UBB} e o {\tt PFS} original.

A próxima etapa do nosso trabalho será o estudo de algoritmos de 
aproximação e do \textit{Probably Approximately
Correct} (PAC learning), que é o modelo de aprendizado no qual se aplica o teorema da navalha de Ockham. Com isso, pretendemos construir uma variante do algoritmo {\tt PFS} que deixa de buscar uma solução ótima para buscar
uma solução provavelmente aproximadamente correta. Após implementarmos o algoritmo de aproximação para o problema U-curve,
poderemos testar o seu desempenho de forma análoga ao que será feito com a versão paralelizada com OpenMP.

Por fim, testaremos todos os algoritmos desenvolvidos durante o projeto
com instâncias artificiais e reais do problema U-curve. Para gerar instâncias artificiais, empregaremos a redução polinomial do problema {\em subset sum} para instâncias do problema U-curve apresentada em Reis~\cite{msreis thesis}, assim como outras funções custo que o beneficiário utilizou durante seu estágio na Universidade Texas A\&M (EUA), nas quais são introduzidas oscilações nas curvas em U das cadeias do reticulado Booleano. Já para instâncias reais, utilizaremos conjuntos de dados próprios para benchmarking de algoritmos, tais como os abrigados no \href{archive.ics.uci.edu/ml}{UCI Machine Learning Repository}.


A organização do plano de trabalho em uma lista de atividades é apresentada na na seção~\ref{sec:atividades}, enquanto que um cronograma para execução dessas atividades é proposto na seção seguinte (tabela~\ref{tab:cronograma}). 

\subsection{Descrição de atividades} \label{sec:atividades}
\begin{itemize}
    \item{\bf Atividade 1.}
        Estudo do algoritmo {\tt PFS}.
    \item{\bf Atividade 2.}
        Implementação de uma variante do {\tt PFS} que usa ROBDDs como 
        estrutura de dados para representar as coleções de raízes das árvores das florestas.
    \item{\bf Atividade 3.}
        Estudo da biblioteca OpenMP, para paralelização do {\tt PFS}.
    \item{\bf Atividade 4.}
        Implementação de uma versão paralela do algoritmo {\tt PFS} com
        o uso de ROBDDs como estrutura de dados.
    \item{\bf Atividade 5.}
        Testes da primeira versão do {\tt PFS} paralelizado com instâncias
        artificiais.
    \item{\bf Atividade 6.}
        Estudos do {\em PAC learning} e do teorema da navalha de Ockham.
    \item{\bf Atividade 7.} 
        Implementação de versões do {\tt PFS}, sequencial e paralelizada, que se 
        comportem como algoritmos de aproximação.
    \item{\bf Atividade 8.}
        Testes com instâncias artificiais dos algoritmos de aproximação.
    \item{\bf Atividade 9.}
        Estudo comparativo entre os algoritmos produzidos, incluindo testes com instâncias artificiais e também de problemas reais.
\end{itemize}

\subsection{Cronograma}

\begin{table}[!ht]
\caption{cronograma de atividades previstas nesta proposta de projeto.} 
\label{tab:cronograma}
\begin{center}
\smallskip
\begin{tabular}{l ccc ccc}
    \toprule
    \small Atividade/mês & \small Jan.17 & \small Fev.17 & \small Mar.17
                         & \small Abr.17 & \small Mai.17 & \small Jun.17
    \\ \hline

    \small Atividade 1   
    & \small {\bf x} & \small - & \small - & \small - & \small - \\

    \small Atividade 2   
    & \small {\bf x} & \small {\bf x} & \small - & \small - 
    & \small - \\

    \small Atividade 3   
    & \small - & \small {\bf x} & \small {\bf x} & \small - & \small -
    & \small - \\

    \small Atividade 4
    & \small - & \small - & \small {\bf x} & \small {\bf x} 
    & \small {\bf x} &  \small - \\

    \small Atividade 5   
    & \small - & \small - & \small - & \small - & \small {\bf x} 
    & \small {\bf x} \\

    \small Primeiro Relatório  
    & \small - & \small - & \small - & \small - & \small {\bf x} 
    & \small {\bf x} \\
    \bottomrule
    \toprule
    \small Atividade/mês & \small Jul.17 & \small Ago.17 & \small Set.17
                         & \small Out.17 & \small Nov.17 & \small Dez.17
    \\ \hline

    \small Atividade 6   
    & \small {\bf x} & \small {\bf x} & \small - & \small - & \small - 
    & \small - \\

    \small Atividade 7
    & \small - & \small {\bf x} & \small {\bf x} & \small{\bf x} 
    & \small - &  \small -\\

    \small Atividade 8  
    & \small - & \small - & \small - & \small {\bf x} & \small {\bf x} 
    & \small - \\

    \small Atividade 9
    & \small - & \small - & \small - & \small - & \small {\bf x} 
    & \small {\bf x}\\

    \small Segundo Relatório
    & \small - & \small - & \small - & \small - & \small {\bf x} 
    & \small {\bf x}\\
    \bottomrule
\end{tabular}
\end{center}
\end{table}


% Arcabouço featsel, já contando com os acréscimos de classes de ROBDDs;
% Biblioteca OpenMP.
\section{Materiais e métodos}
Os algoritmos produzidos serão implementados no arcabouço \href{https://github.com/msreis/featsel}{featsel}. Esse arcabouço foi implementado em C++ e conta com diversas estruturas de dados que serão necessárias para implementar os
algoritmos sugeridos nesse trabalho. Além disso, featsel já conta com classes que implementam ROBDDs; tal melhoramento foi feito durante a Iniciação Científica anterior do beneficiário, numa tentativa de melhorar o desempenho do algoritmo U-Curve-Search ({\tt UCS}), resultando em ganhos de desempenho do ponto de vista de consumo de tempo computacional~\cite{ucsrobdd ic}. O arcabouço featsel está disponível sob a licença \textit{GNU General Public License} (GNU-GPL), o que nos permite utilizar livremente o código desse arcabouço para implementação dos novos algoritmos. 

Também utilizaremos a biblioteca \href{http://www.openmp.org/}{OpenMP} junto ao compilador {\tt g++} para compilar os códigos produzidos. Por meio de diretivas de compilação, OpenMP determina como um código pode ser executado em paralelo no processador.

Experimentos computacionais, em particular os que envolverem paralelização de algoritmos, serão feitos em uma servidora do Instituto Butantan que é própria para esse fim: uma Dell PowerEdge 850, com 64 núcleos e 256 GB de memória RAM.


% Benchmarking contra outros algoritmos de seleção de características;
% Elaboração de paper para ser enviado para publicação ao final da IC
% proposta.
\section{Forma de Análise e de Divulgação dos Resultados}

Para analisar os resultados obtidos utilizaremos o próprio arcabouço \href{https://github.com/msreis/featsel}{featsel}, que já possui métodos que registram dados sobre a execução dos algoritmos, assim como um programa auxiliar para {\em benchmarking} de algoritmos e de funções custo implementados nessa ferramenta. As principais métricas a serem consideradas nos experimentos serão tempo de execução, consumo de memória e a robustez dos algoritmos quando violamos a suposição de que a função custo é decomponível em curvas em U. O código-fonte final do projeto será disponibilizado para a comunidade acadêmica no \href{https://github.com/msreis/featsel}{repositório web do arcabouço featsel}.

Além disso, ao longo de 2017, o beneficiário deverá se matricular em \href{https://uspdigital.usp.br/jupiterweb/obterDisciplina?sgldis=MAC0499\&nomdis=}{MAC499 -- Trabalho de Formatura Supervisionado}, disciplina do IME-USP que exige a preparação de uma monografia ao final do curso e também a apresentação de um pôster contendo resultados finais da Iniciação Científica.

Por fim, projetamos a escrita de um artigo científico, no qual descreveremos os avanços que os novos algoritmos trouxeram, além de eventuais propriedades do problema U-curve e de nossa solução que 
possam ser exploradas futuramente para elaboração de novas abordagens para resolver esse importante problema de otimização.

\begin{thebibliography}{}
\addcontentsline{toc}{section}{Referências}
\bibitem{msreis thesis}
    Marcelo S. Reis. ``Minimização de funções decomponíveis em curvas em U definidas sobre cadeias de posets -- algoritmos e aplicações".
    Tese de doutorado. Instituto de Matemática e Estatística, Universidade de São Paulo, Brasil (2012).

\bibitem{bryant}
Randal E. Bryant. ``Graph-based algorithms for boolean function manipulation". IEEE Transactions on Computers, v. 100.8, p. 677--691 (1986). 

\bibitem{kearns}
Michael J. Kearns e Umesh V. Vazirani. ``An introduction to computational learning theory". MIT Press (1994).

\bibitem{ucsrobdd ic}
Gustavo E. Matos e Marcelo S. Reis. ``Estudos de estruturas de dados eficientes para abordar o problema de otimização U-curve". Relatório científico final FAPESP, Instituto Butantan, Brasil (2015).



    %% como citar ic?

\end{thebibliography}
\end{document}
