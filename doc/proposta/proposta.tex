\documentclass[12pt]{article}
\usepackage[portuguese]{babel}
\usepackage[utf8]{inputenc}
\usepackage[usenames,dvipsnames]{color}
\usepackage{setspace}
\usepackage{amsmath}
\usepackage{amsfonts}
\usepackage{amssymb}
\usepackage{mathtools}
\usepackage[top=3cm, bottom=2cm, left=3cm, right=2cm]{geometry}
\usepackage{tikz}
\usepackage{textcomp}
\usepackage{lscape}    % for landscape pages
\usepackage{hyperref}  % to allow hyperlinks
\usepackage{booktabs}  % nicer table borders
\usepackage{subfigure} % add subfigures

\title{Projeto IC PFS}

% Figures directory
\graphicspath{{./figures/}} 

\definecolor{myblue}{RGB}{80,80,160}
\definecolor{mygreen}{RGB}{80,160,80}
\setstretch{1.5}

\begin{document}

% FAPESP demands the usage of double spacing
%
\doublespacing

\begin{center}
    {\LARGE Projeto de algoritmos baseados em florestas de posets\\
        \bigskip 
        para o problema de otimização U-curve}

    \bigskip        

    {\large {\bf Aluno:} \href{mailto:gustavo.estrela.matos@usp.br}
        {Gustavo Estrela de Matos}\\ 
    {\bf Orientador:} \href{mailto:marcelo.reis@butantan.gov.br}
        {Marcelo da Silva Reis}\\

    \bigskip

    \today\\
    }

    \bigskip
    \bigskip

    {\bf Resumo}    
\end{center}
    % Resumo

\newpage

\begin{center}
    {\LARGE Design of poset forest-based algorithms for the\\
        \bigskip 
        U-curve optimization problem}

    \bigskip        

    {\large {\bf Student:} \href{mailto:gustavo.estrela.matos@usp.br}
        {Gustavo Estrela de Matos}\\ 
    {\bf Supervisor:} \href{mailto:marcelo.reis@butantan.gov.br}
        {Marcelo da Silva Reis}\\

    \bigskip

    \today\\
    }

    \bigskip
    \bigskip

    {\bf Abstract}    
\end{center}
    % Resumo
    
\newpage
\tableofcontents
\newpage

\section{Introdução}
% Apresentação do problema U-curve (na linha do projeto da primeira IC,
% só que procurando ser mais sucinto);
% Recapitulação da IC anterior (melhoramentos do algoritmo UCS com 
% ROBDDs), com particular destaque para a limitação dos melhoramentos 
% obtidos.
% 
\subsection{O problema U-Curve}
O problema de seleção de característica consiste em, dado um cojunto $S$
de características, escolher um subconjunto de características que seja
ótimo. A solução desse problema tem aplicações na construção de modelos
para aprendizado de máquina e reconhecimento de padrões, que dependem da
escolha de um subconjunto de características que seja o mais relevante 
possível (ótimo), de acordo com alguma métrica. Formalmente, podemos 
definir o problema de seleção de características como um problema de 
busca, no qual procuramos um subconjunto $X \in \mathcal{P}(S)$ que 
minimiza uma função de custo $c : \mathcal{P}(S) \to \mathbb{R_+}$.

O espaço de busca do problema de seleção de características pode ser
visto como um reticulado booleano ($\mathcal{P}(S)$, $\subseteq$), onde
cada nó é um conjunto de características, também chamado de 
classificador. É comum nesse problema que as cadeias do reticulado
descrevam "curvas em u" quando avaliadas pela  função de custo $c$. Esse
comportamento pode ser intuitivamente explicado se considerarmos que um 
classificador melhora ao adicionarmos novas características até um ponto
em que o grande número de características causa grandes erros de 
estimação, piorando o classificador. 

O problema U-Curve é um caso particular do problema de seleção de
características em que todas as cadeias do espaço de busca descrevem
"curvas em u" quando avaliadas pela função de custo. Existem algoritmos
ótimos para solução do problema U-Curve, como o Poset-Forest Search
({\tt PFS}) e U-Curve Search ({\tt UCS}) ~\cite{msreis thesis}. Além 
disso, em outra oportunidade de iniciação científica, estudamos o uso
de diagramas de decisão binários ordenados e reduzidos (ROBDDs) como uma
estrutura de dados eficiente para o controle do espaço de busca 
~\cite{ucsrobdd ic}.

O uso de ROBDDs aliado a mudanças à dinâmica do {\tt UCS} nos levaram a 
criação do algoritmo {\tt UCSR}. Esse novo algoritmo trouxe melhoras no
tempo de execução, pois permite consultas rápidas ao espaço de busca, o
que era mais custoso no algoritmo {\tt UCS}. Porém, as melhorias obtidas
foram limitadas, uma vez que manter a estrutura de ROBDD, em alguns 
casos, demandava grande processamento e uso de memória. Portanto, 
torna-se necessário a criação de novos algoritmos para resolver o 
problema, o que nos leva ao estudo do algoritmo Poset-Fores Search 
({\tt PFS}).


% Levando-se em consideração tais limitações, existe a necessidade de
% novos algoritmos para atacar esse problema. Nesse sentido, foi 
% proposto o algoritmo Poset-Forest Search (PFS) (Reis, 2012, capítulo 
% 6.2 ~\cite{msreis thesis}). Uma versão preliminar foi implementada e
% testada, com resultados promissores; todavia, novos melhoramentos e
% testes são necessários para que o PFS se torne uma alternativa
% competitiva para resolver o problema de seleção de características.
\subsection{O algoritmo Poset-Forest Search}

\section{Objetivos}

\begin{enumerate}
\item Utilização dos ROBDDs, implementados no IC anterior, para representar as 
listas de raízes do algoritmo PFS.

\item Desenho de uma versão paralelizada do PFS, com maior escalabilidade. Para
este fim, paralelizaremos o percorrimento das florestas de posets, com o
programa principal gerenciando a escolha das raízes (i.e., início de um percor-
rimento), guardando o mínimo corrente e centralizando a atualização das podas.

\item Desenvolvimento de uma versão do PFS que funcione como algoritmo de 
aproximação para o problema U-curve, utilizando como critério de aproximação da
solução ótima o teorema da navalha de Ockham:\\
\smallskip
Dado um espaço de hipóteses H (i.e., espaço de busca), o número mínimo de
amostras necessário para se obter uma solução que erra no máximo $\epsilon$ com
$1 - \delta$ de probabilidade é expresso por:
\begin{equation}
\displaystyle  m(\delta,\epsilon) = \frac{1}{\epsilon} log (\frac{|H|}{\delta}).
\end{equation}

\item Implementação e testes dos algoritmos propostos, para isso empregando o 
arcabouço featsel.
\end{enumerate}


\section{Plano de trabalho}

\subsection{Cronograma}

Tabela listando atividades de janeiro a dezembro de 2017.

\subsection{Descrição de atividades}

Descrição das atividades da tabela da subseção anterior.

\section{Materiais e métodos}

\section{Forma de análise e divulgação de resultados}

\begin{itemize}
\item Arcabouço featsel, já contando com os acréscimos de classes de ROBDDs;

\item Biblioteca OpenMP.
\end{itemize}

\section{Forma de Análise e de Divulgação dos Resultados}

\begin{itemize}
\item Benchmarking contra outros algoritmos de seleção de caracteríscas;

\item Elaboração de paper para ser enviado para publicação ao final da IC proposta.
\end{itemize}

\newpage
\begin{thebibliography}{}
\addcontentsline{toc}{section}{Referências}
\bibitem{msreis thesis}
    Reis, Marcelo S. "Minimization of decomposable in U-shaped curves 
    functions defined on poset chains–algorithms and applications."
    PhD thesis, Institute of Mathematics and Statistics, University of
    São Paulo, Brazil, (2012).

\bibitem{ucsrobdd ic}
    %% como citar ic?

\end{thebibliography}
\end{document}
